% Created 2012-10-30 Tue 13:07
\documentclass[11pt]{article}
\usepackage[utf8]{inputenc}
\usepackage[T1]{fontenc}
\usepackage{fixltx2e}
\usepackage{graphicx}
\usepackage{longtable}
\usepackage{float}
\usepackage{wrapfig}
\usepackage{soul}
\usepackage{textcomp}
\usepackage{marvosym}
\usepackage{wasysym}
\usepackage{latexsym}
\usepackage{amssymb}
\usepackage{hyperref}
\tolerance=1000
\providecommand{\alert}[1]{\textbf{#1}}

\title{Riittävä suorituskyky on keskeinen laatuvaatimus mille tahansa}
\author{Timo I Tuominen}
\date{\today}
\hypersetup{
  pdfkeywords={},
  pdfsubject={},
  pdfcreator={Emacs Org-mode version 7.8.11}}

\begin{document}

\maketitle

% Org-mode is exporting headings to 3 levels.

ohjelmistolle, mutta millä tavalla suorituskykyvaatimukset tulisi
ottaa huomioon ohjelmiston kehitysprosessissa? Ohjelmistotekniikan
kirjallisuudessa kysymykseen on vastattu hyvin eri tavoin.

Donald Knuthilta on peräisin tunnettu lainaus: ``Ennenaikainen
optimointi on kaiken pahan alku ja juuri.''\cite{todo} Käsitellessään \textbf{rakenteellisen ohjelmoinnin periaatteita} Knuth korostoaa, ettei
varsinaisessa ohjelmointityössä kannata käyttää aikaa suorituskykyä
parantaviin manuaalisiin optimointeihin ennen kuin ohjelman
suoritusaikaista käyttäytymistä on tutkittu ja löydetty ne ohjelman
osat, joiden suoritusaika on merkittävä osa ohjelman
kokonaissuoritusajasta.

Periaatteeseen on kaksi syytä. Ensinnäkin ohjelmoijan intuitio johtaa
yleensä harhaan, kun etsitään ohjelman suorituskyvyn kannalta
keskeisiä osia. Lisäksi käsityönä tehdyt optimoinnit tekevät
ohjelmakoodista vaikeammin ymmärrettävää ja ylläpidettävää, joten
niitä kannattaa välttää kaikkialla, missä niitä ei todistetusti
tarvita.

Knuthin tueksi voi nykynäkökulmasta todeta, että
laitteistoarkkitehtuurien, optimoivien kääntäjien, virtuaalikoneiden
ja suoritusympäristön yleisen monimutkaistumisen takia pelkästä
ohjelman lähdekoodista on yhä vaikeampaa suoraan päätellä ohjelman
suoritusaikaista käyttäytymistä.

\section{\textbf{TODO} esimerkki lähteistä}
\label{sec-1}


Knuthin näkökulma rajottui kuitenkin yksittäisen algoritmin, ohjelman
tai komponentin ohjelmointiin. Laaja

\end{document}
